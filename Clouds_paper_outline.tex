\documentclass[a4,useAMS,usenatbib]{mn2e}

\usepackage{color}
\usepackage{epsfig}
\usepackage{rotating}
\usepackage{multirow}
\usepackage{amsmath}
\usepackage{amssymb}

%%%%% ASTRO-PH formatting
\setlength{\topmargin}{-0.625in}
\setlength{\oddsidemargin}{-0.25in}
\setlength{\evensidemargin}{-0.25in}

\newcommand*{\chck}[1]{{\color{red}$<$#1$>$}}

\def\degree{$^{\circ}$}
\def\herschel{{\em Herschel}}
\def\spire{\herschel-SPIRE}
\def\pacs{\herschel-PACS}
\def\spitzer{{\em Spitzer}}
\def\mic{$\umu$m}

\title{Clouds Paper}

\author[The world]{R.~Simpson$^1$, c.~J.~Lintott$^1$, C.~E.~North$^2$ et al.
\\
$^1$Oxford
$^2$Cardiff
}
\begin{document}

\date{Accepted 20xx. Received 20xx.}

\pagerange{\pageref{firstpage}--\pageref{lastpage}} \pubyear{2012}

\maketitle

\label{firstpage}

\begin{abstract}
Abstract
\end{abstract}

\section{Introduction}
Infrared dark clouds (IRDCs) were first observed as dark regions
silhouetted against the mid-infrared (MIR) background
\citep{Wilcock2011}. Subsequent observations showed them to have low
temperatures and high densities ($T\lesssim$\,K, $n_H > 10^5$\,cm$^-3$,
e.g.~\citet{Egan98,Carey98,HennebellePerault02}). The IRDC absorbs the
background light and causes a dip in the MIR sky brightness. They are
thought to be the earliest observable formation stages high-mass stars
and stellar clusters.

With MIR data alone, however, it is impossible to distinguish this
absorption from a region of inherently lower background
emission. Far-Infared (FIR) observations allow IRDCs, which appear
bright at wavelengths above 24\,\mic, to be distinguished from regions
of lower emission, which remain dark. 


\section{Input catalogue}
Describe catalogue (Fuller \& Peretto)
Spitzer \& Herschel Hi-Gal observations

Number of objects etc.

\section{Analysis}
Brief outline of analysis

\subsection{Milky Way Project}
The Milky Way Project\footnote{http://www.milkywayproject.org} was established in 2010 as a citizen science interface to data from the \emph{Spitzer} GLIMPSE survey primarily as a search for `bubbles' associated with massive star formation. This effort was successful, and a catalogue of more than 5000 such bubbles which expanded on previous efforts by professional astronomers was published by \citet{Simpsonetal} and used for a statistical analysis of bubble distribution by \citet{Kendrewetal}. Inspired by this success, a second interface was added to the site in order to address the problem of identifying true IRDCs.

As with the previous interface, this new part of the site\footnote{http://www.milkywayproject.org/clouds} makes use of the Zooniverse Application Programming Interface (API) originally built for Galaxy Zoo \citep{Lintottetal} and which supports a large number of similar citizen science projects. This API is primarily responsible for serving images and recording classifications provided by volunteers, who are required to be logged in for their work to be recorded. Following a short tutorial, an image is selected from the database\footnote{Volunteers see an image they have not yet classified, selected randomly from those with the fewest classifications in the database. This algorithm for task assignment has the advantage of ensuring that images have approximately the same number of classifications at all times, facilitating preliminary data analysis.} and presented to the volunteer who may label it as a \textsc{cloud}, a \textsc{hole} or an \textsc{intermediate} case by selecting one of three buttons. 

MWP interface and initial results (or in results section?)
User and classification numbers and duration of dataset used

Raw classifications?
Histogram of results?

Thumnail examples of clouds, holes and unknowns

\subsection{Experts and Training data}
Definition of training data definition

Thumbnails of training data and classifications

Expert versus general results (plot)

\subsection{Analysis sequence}
The procedure
Growth charts
Results of MC runs
Any threshholds


\section{Results}
Results from analyis (cloudiness chart)

Histogram of classifications before \& after

Thumnail examples of clouds, holes and unknowns

Full table of results

\section{Conclusions}


\section{Acknowledgements}
\chck{Zooniverse acknowledgements}

{\em Herschel} is an ESA space observatory with science instruments
provided by European-led Principal Investigator consortia and with important
participation from NASA.

SPIRE has been developed by a consortium of institutes led by Cardiff
Univ. (UK) and including: Univ. Lethbridge (Canada); NAOC (China);
CEA, LAM (France); IFSI, Univ. Padua (Italy); IAC (Spain); Stockholm
Observatory (Sweden); Imperial College London, RAL, UCL-MSSL, UKATC,
Univ. Sussex (UK); and Caltech, JPL, NHSC, Univ. Colorado (USA). This
development has been supported by national funding agencies: CSA
(Canada); NAOC (China); CEA, CNES, CNRS (France); ASI (Italy); MCINN
(Spain); SNSB (Sweden); STFC, UKSA (UK); and NASA (USA).

This research made use of APLpy, an open-source plotting package for
Python hosted at http://aplpy.github.com

\end{document}
